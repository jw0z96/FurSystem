%%% template.tex
%%%
%%% This LaTeX source document can be used as the basis for your technical
%%% paper or abstract. Intentionally stripped of annotation, the parameters
%%% and commands should be adjusted for your particular paper - title,
%%% author, article DOI, etc.
%%% The accompanying ``template.annotated.tex'' provides copious annotation
%%% for the commands and parameters found in the source document. (The code
%%% is identical in ``template.tex'' and ``template.annotated.tex.'')

\documentclass[]{acmsiggraph}
\usepackage{algorithm}
\usepackage[noend]{algpseudocode}
\TOGonlineid{45678}
\TOGvolume{0}
\TOGnumber{0}
\TOGarticleDOI{0}
\TOGprojectURL{}
\TOGvideoURL{}
\TOGdataURL{}
\TOGcodeURL{}
\usepackage{color}
%\definecolor{red}{rgb}{0.9, 0.17, 0.31}
\usepackage{multirow}
\usepackage{subfig}
\usepackage{xcolor}
\usepackage{lipsum}
\usepackage{listings}
\usepackage{graphicx}
\usepackage{glsllst} % My own package providing markup listing for glsl
\usepackage{rmlst}   % My own package providing markup listing for renderman
\usepackage{amsmath}
\usepackage{hyperref}

\lstset{
	backgroundcolor=\color[rgb]{0.95, 0.95, 0.95},
	tabsize=3,
	%rulecolor=,
	basicstyle=\footnotesize\ttfamily,
	upquote=true,
	aboveskip={1.5\baselineskip},
	columns=fixed,
	showstringspaces=false,
	extendedchars=true,
	breaklines=true,
	prebreak = \raisebox{0ex}[0ex][0ex]{\ensuremath{\hookleftarrow}},
	frame=none,
	aboveskip=15pt,
	belowskip=8pt,
	captionpos=t,
	showtabs=false,
	showspaces=false,
	showstringspaces=false,
	identifierstyle=\ttfamily,
	%keywordstyle=\color{red}\bfseries,
	%keywordstyle=[1]\bfseries\color{syntaxBlue},
	%keywordstyle=[2]\bfseries\color{syntaxRed},
	%keywordstyle=[3]\color{blue}\bfseries,
	%keywordstyle=[4]\bfseries\color{syntaxBlue},
	commentstyle=\color[rgb]{0.082,0.639,0.082},
	keywordstyle=[1]\bfseries\color[rgb]{0,0,0.75},
	keywordstyle=[2]\bfseries\color[rgb]{0.5,0.0,0.0},
	keywordstyle=[3]\bfseries\color[rgb]{0.127,0.427,0.514},
	keywordstyle=[4]\bfseries\color[rgb]{0.4,0.4,0.4},
	stringstyle=\color[rgb]{0.639,0.082,0.082},
}

\title{Innovations Report: Procedural Fur System}

\author{Joe Withers\thanks{e-mail:joewithers96@gmail.com}\\National Centre for Computer Animation}
\pdfauthor{Joe Withers}

\keywords{rendering}

\begin{document}

\maketitle

\begin{abstract}
For this project I developed a procedural fur system, with the aim of exploring the feasibility of offloading computation onto the GPU within artist tools. The final artefact is an application which serves to interface with the fur system API that I have developed. This report primarily documents the implementation of the API, as well as my findings.
\end{abstract}

\section{Introduction} \label{sec:introduction}
Introduction Text.

\subsection{Existing Solutions} \label{sec:existing}
Existing Solutions. (Xgen, houdini)

\section{Implementation} \label{sec:implementation}
Method Text.

\subsection{Resources} \label{sec:resources}
I decided to develop the API side of the fur system using C++, primarly as it most commonly used when developing computationally heavy artist tools, but also because it is the language I am most comfortable using. I opted to use OpenGL over other APIs (CUDA, OpenCL) for offloading of computation onto the GPU, simply because my final artefact is a tool with a graphical interface, and OpenGL is capable of handling both arbitary computation using compute shaders and rendering of geometry.

To handle the user interface I used the Qt framework within C++. Qt is commonly used for artist tools within visual effects as it is cross-platform, and applications can be configured to run within other applications that make use of it, such as Autodesk Maya.

I wanted to include a node-graph style interface within my application, as it is commonly used within existing artist tools (Autodesk Maya, Unreal Engine), and would encourage modularity within my API. Qt does not natively provide this kind of interface, so I made use of NodeEditor \cite{Pinaev2017} an existing Qt-based library that provides this functionality.

\subsection{Design} \label{sec:design}
From looking at existing fur systems I was able determine that my simplified fur system would need to consist of the following components:
\begin{itemize}
	\item Distributors - These are responsible for the distribution of curves onto user specified geometry. User controllable parameters could include density (curve count), distribution pattern (random, uniform), and curve length. These parameters could potentially be controlled by texture inputs.
	\item Operators - These are responsible for manipulation of the curves to achieve the desired look. Example operators could provide bending, clumping, or randomisation of input curves.
	\item Renderers - These are responsible for the rendering of curves into the application viewport. Example renderers could provide mesh, cuves as lines, or curves as ribbons rendering functionality. User controllable parameters could provide controls for the shading model in use, as well as control of the base and tip widths when rendering curve ribbons for example.
\end{itemize}

\section{Results} \label{sec:results}
Results Text.

\bibliographystyle{acmsiggraph}
\bibliography{references}

\end{document}
